%Einbindekorrektur als optionales Argument "`BCORfaktormitmaßeinheit"', dann
% sieht auch Option "`twoside"' vernünftig aus
% Näheres zu "`scrartcl"' bzw. "`scrreprt"' und "`scrbook"' siehe KOMA-Skript Doku
\documentclass[12pt,a4paper,titlepage,headinclude,bibtotoc]{scrartcl}


%---- Allgemeine Layout Einstellungen ------------------------------------------

% Für Kopf und Fußzeilen, siehe auch KOMA-Skript Doku
\usepackage[komastyle]{scrpage2}
\pagestyle{scrheadings}
\setheadsepline{0.5pt}[\color{black}]
\automark[section]{chapter}


%Einstellungen für Figuren- und Tabellenbeschriftungen
\setkomafont{captionlabel}{\sffamily\bfseries}
\setcapindent{0em}


%---- Weitere Pakete -----------------------------------------------------------
% Die Pakete sind alle in der TeX Live Distribution enthalten. Wichtige Adressen
% www.ctan.org, www.dante.de

% Sprachunterstützung
\usepackage[ngerman]{babel}

% Benutzung von Umlauten direkt im Text
% entweder "`latin1"' oder "`utf8"'
\usepackage[utf8]{inputenc}

% Pakete mit Mathesymbolen und zur Beseitigung von Schwächen der Mathe-Umgebung
\usepackage{latexsym,exscale,stmaryrd,amssymb,amsmath}

% Weitere Symbole
\usepackage[nointegrals]{wasysym}
\usepackage{eurosym}

% Anderes Literaturverzeichnisformat
%\usepackage[square,sort&compress]{natbib}

% Für Farbe
\usepackage{color}

% Zur Graphikausgabe
%Beipiel: \includegraphics[width=\textwidth]{grafik.png}
\usepackage{graphicx}

% Text umfließt Graphiken und Tabellen
% Beispiel:
% \begin{wrapfigure}[Zeilenanzahl]{"`l"' oder "`r"'}{breite}
%   \centering
%   \includegraphics[width=\dots]{grafik}
%   \caption{Beschriftung} 
%   \label{fig:grafik}
% \end{wrapfigure}
\usepackage{wrapfig}

% Mehrere Abbildungen nebeneinander
% Beispiel:
% \begin{figure}[htb]
%   \centering
%   \subfigure[Beschriftung 1\label{fig:label1}]
%   {\includegraphics[width=0.49\textwidth]{grafik1}}
%   \hfill
%   \subfigure[Beschriftung 2\label{fig:label2}]
%   {\includegraphics[width=0.49\textwidth]{grafik2}}
%   \caption{Beschriftung allgemein}
%   \label{fig:label-gesamt}
% \end{figure}
\usepackage{subfigure}

% Caption neben Abbildung
% Beispiel:
% \sidecaptionvpos{figure}{"`c"' oder "`t"' oder "`b"'}
% \begin{SCfigure}[rel. Breite (normalerweise = 1)][hbt]
%   \centering
%   \includegraphics[width=0.5\textwidth]{grafik.png}
%   \caption{Beschreibung}
%   \label{fig:}
% \end{SCfigure}
\usepackage{sidecap}

% Befehl für "`Entspricht"'-Zeichen
\newcommand{\corresponds}{\ensuremath{\mathrel{\widehat{=}}}}
% Befehl für Errorfunction
\newcommand{\erf}[1]{\text{ erf}\ensuremath{\left( #1 \right)}}

%Fußnoten zwingend auf diese Seite setzen
\interfootnotelinepenalty=1000

%Für chemische Formeln (von www.dante.de)
%% Anpassung an LaTeX(2e) von Bernd Raichle
\makeatletter
\DeclareRobustCommand{\chemical}[1]{%
{\(\m@th
\edef\resetfontdimens{\noexpand\)%
\fontdimen16\textfont2=\the\fontdimen16\textfont2
\fontdimen17\textfont2=\the\fontdimen17\textfont2\relax}%
\fontdimen16\textfont2=2.7pt \fontdimen17\textfont2=2.7pt
\mathrm{#1}%
\resetfontdimens}}
\makeatother

%Honecker-Kasten mit $\shadowbox{$xxxx$}$
\usepackage{fancybox}
%Honeckerbox mit Mathe \begin{emphaeq}[box=\shadowbox*]{align}\dots\end{emphaeq}
%\usepackage{emphaeq}
						    
%SI-Package
\usepackage{siunitx}


\parindent0pt

%Bibliography \bibliography{literatur} und \cite{gerthsen}
%\usepackage{cite}
\usepackage{babelbib}
\selectbiblanguage{ngerman}

\begin{document}

\begin{titlepage}
\centering
\textsc{\Large Anfängerpraktikum der Fakultät für
  Physik,\\[1.5ex] Universität Göttingen}

\vspace*{4.2cm}

\rule{\textwidth}{1pt}\\[0.5cm]
{\huge \bfseries
  Versuch Adiabatenexponent\\[1.5ex]
  Protokoll}\\[0.5cm]
\rule{\textwidth}{1pt}

\vspace*{3cm}

\begin{Large}
\begin{tabular}{ll}
Praktikant: 	& Michael Lohmann\\
% 		& Kevin Lüdemann\\
		& Skrollan Detzler\\
E-Mail: 	& m.lohmann@stud.uni-goettingen.de\\
 		& skrollan.detzler@stud.uni-goettingen.de\\
% 		& kevin.luedemann@stud.uni-goettingen.de\\
Versuchsdatum:	& 16.6.2014\\
Betreuer: 	& Martin Ochmann\\
\end{tabular}
\end{Large}

\vspace*{0.8cm}

\begin{Large}
\fbox{
  \begin{minipage}[t][2.5cm][t]{6cm} 
    Testat:
  \end{minipage}
}
\end{Large}

\end{titlepage}

\tableofcontents

\newpage

\section{Einleitung}
\label{sec:einleitung}
Der Adiabatenexponent ist ein wichtiges Kennzeichen von Gasen.
Er beschreibt das Verhältnis des Wärmespeicherkoeffizienten bei konstantem Druck zu dem mit konstantem Volumen (\cite[S. 263]{gerthsen}).
In der Regel wird er mit $\kappa$ bezeichnet.

\section{Theorie}
\label{sec:theorie}
                                                                                                                                                                      

\section{Durchführung}
\label{sec:durchfuehrung}


\section{Auswertung}
\label{sec:auswertung}

\subsection{Messung nach Rüchard}
Die aufbauspezifischen Daten unseres Versuchs lauten:
\begin{table}[!h]
	\centering
	\begin{tabular}{|l|l|}
		\hline
		Messgröße	& Messwert\\\hline\hline
		Masse		& $m=4.88$ g\\\hline
		Durchmesser	& $d=9.97$ mm\\\hline
		Volumen		& $V=2300.45$ cm$^3$\\\hline
		Luftdruck	& $b_1=1015.8$ hPa\\
		- nachher	& $b_2=1015.5$ hPa\\\hline
		Temperatur	& $T_1=25.9^\circ$ C\\
		- nachher	& $T_2=23.6^\circ$ C\\\hline
	\end{tabular}
	\caption{Versuchsspezifische Größen}
	\label{tab:versgr}
\end{table}
Da beim schwingenden Gewicht in der Röhre zusätzlich noch das sich darin befindliche Gas bewegt werden muss, ist die effektive Masse $m_\text{eff}$ höher:
\begin{align*}
	m_\text{eff}&=m+\rho_L \cdot A\cdot l\\
	\sigma_{m_\text{eff}}&=\sigma_l\cdot \rho_l \cdot A
\end{align*}
Der daraus resultierende Druck $p$ wird durch
\begin{align*}
	p&=b+\frac{m_\text{eff}~g}{A}\\
	\sigma_p&=\sqrt{\sigma_b^2+\sigma_{m_\text{eff}}^2\left(\frac{g}{A}\right)^2}
\end{align*}
berechnet.
Die Werte für unseren Versuch sind in Tabelle \ref{tab:effm} dargestellt.
\begin{table}[!htbp]
	\centering
	\begin{tabular}{|l|c|c|}
		\hline
		Gas	& $m_\text{eff}$ [g]	& $p$ [hPa]\\\hline\hline
		CO$_2$	& $4.8983 \pm 0.0005$ 	& $1021.81 \pm 0.10$ \\\hline
		Argon	& $4.8917 \pm 0.0005$ 	& $1021.80 \pm 0.10$ \\\hline
		Luft	& $4.8964 \pm 0.0005$ 	& $1021.80 \pm 0.10$ \\\hline
	\end{tabular}
	\caption{Effektive Masse zu den einzelnen Gasen und die daraus resultierenden Drücke} 
	\label{tab:effm}
\end{table}

\begin{align*}
	\kappa&=\frac{4\pi^2 \cdot m_{\text{eff}}\cdot V }{T^{2} \cdot p \cdot d^{4}}\\
	\sigma_{\kappa}&=\frac{4\pi^2 ~ V}{T^{3}  d^{4}  p^{2}} \cdot \sqrt{\left(T  m_{\text{eff}}\right)^2 \cdot \sigma_{p}^{2} + \left(T  p\right)^2 \cdot \sigma_{m_{\text{eff}}}^{2} + \left(2m_{\text{eff}}~p\right)^{2} \cdot \sigma_{T}^{2}}
\end{align*}

\begin{table}
	\centering
	\begin{tabular}{|l|l|l|l|}
		\hline
		Gas 	&Schwingungen & Periodendauer [ms] & $\kappa$ \\\hline\hline

			& 1 	& $762.1 \pm 1.1$	& $0.7587 \pm 0.0021$ \\
			& 10 	& $762.23 \pm 0.24$ 	& $0.7584 \pm 0.0005$ \\
		CO$_2$	& 20 	& $763.29 \pm 0.11$ 	& $0.75629 \pm 0.00025$ \\
			& 50 	& $763.39 \pm 0.12$ 	& $0.75610 \pm 0.00026$ \\
			& 100 	& $762.70 \pm 0.22$	& $0.7575 \pm 0.0004$ \\\hline
	
			& 1 	& $685.8 \pm 1.0$	& $0.9356 \pm 0.0028$ \\
			& 10	& $686.5 \pm 0.4$	& $0.9338 \pm 0.0012$ \\
		Argon	& 20	& $686.48 \pm 0.27$	& $0.9337 \pm 0.0008$ \\
			& 50	& $686.48 \pm 0.15$	& $0.9338 \pm 0.0004$ \\
			& 100	& $686.33 \pm 0.06$	& $0.93416 \pm 0.00021$ \\\hline
	
			& 1	& $737.4 \pm 1.0$	& $0.8100 \pm 0.0023$ \\
			& 10	& $737.4 \pm 0.4$	& $0.8101 \pm 0.0008$ \\
		Luft	& 20	& $737.96 \pm 0.25$	& $0.8088 \pm 0.0006$ \\
			& 50	& $738.6 \pm 0.5$	& $0.8074 \pm 0.0012$ \\
			& 100	& $739.1 \pm 0.5$	& $0.8063 \pm 0.0012$ \\\hline
	\end{tabular}
	\caption{Schwingungszeiten unterschiedlicher Gase und die resultierenden $\kappa$}
	\label{tab:schwingungszeit}
\end{table}

\begin{table}[!htb]
	\centering
	\begin{tabular}{|c|c|}
		\hline
		Öffnungszeit [s] & $\kappa$\\\hline
		0.1 & $1.130 \pm 0.014$\\
		1.0 & $1.133 \pm 0.013$\\
		5.0 & $1.106 \pm 0.014$\\
		\hline
	\end{tabular}
	\caption{Gew. Mittelwerte von $\kappa$ zu den jeweiligen Öffnungszeiten}
	\label{tab:kappaclement}
\end{table}

\begin{align*}
	\kappa&=\frac{\Delta h_{1}}{\Delta h_{1} - \Delta h_{2}}\\
	\sigma_{\kappa}&=\frac{1}{\left(\Delta h_{1} - \Delta h_{2}\right)^{2}} \cdot \sqrt{\Delta h_{1}^{2} \cdot \sigma_{\Delta h_2}^{2} + \Delta h_{2}^{2} \cdot \sigma_{\Delta h_1}^{2}}
\end{align*}

\subsection{Messung nach Clement-Desormes}



\section{Diskussion}
\label{sec:diskussion}
In der Tabelle der versuchsspezifischen Größen \ref{tab:versgr} fällt auf, dass sich die Temperatur im Versuchsraum während der Messungen um über $2^\circ$ C geändert hat.
Dies verfälscht die Messwerte, so dass für zukünftige Messungen empfehlenswert ist, zumindest die Fenster zu schließen, so unangenehm dies auch ist.
Noch besser wäre allerdings ein klimatisierter Raum.


\bibliography{literatur}
\bibliographystyle{babalpha}

\end{document}
